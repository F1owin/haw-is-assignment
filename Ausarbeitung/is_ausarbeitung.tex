\documentclass[a4paper, 11pt]{scrartcl}

\usepackage[ngerman]{babel}

\usepackage[utf8]{inputenc}

\usepackage[T1]{fontenc}

\usepackage{ae,aecompl}

\usepackage{amsmath,amssymb,amstext}

\usepackage{psfrag}

\usepackage{biblatex}

\bibliography{library}

\usepackage[automark]{scrpage2}

\usepackage{ifpdf}

\ifpdf%   (definitions for using pdflatex instead of latex)

  %%% graphicx: support for graphics
  \usepackage[pdftex]{graphicx}

  \pdfcompresslevel=9

  %%% hyperref (hyperlinks in PDF): for more options or more detailed
  %%%          explanations, see the documentation of the hyperref-package
  \usepackage[%
    %%% general options
    pdftex=true,      %% sets up hyperref for use with the pdftex program
    %plainpages=false, %% set it to false, if pdflatex complains: ``destination with same identifier already exists''
    %
    %%% extension options
    backref,      %% adds a backlink text to the end of each item in the bibliography
    pagebackref=false, %% if true, creates backward references as a list of page numbers in the bibliography
    colorlinks=true,   %% turn on colored links (true is better for on-screen reading, false is better for printout versions)
    %
    %%% PDF-specific display options
    bookmarks=true,          %% if true, generate PDF bookmarks (requires two passes of pdflatex)
    bookmarksopen=false,     %% if true, show all PDF bookmarks expanded
    bookmarksnumbered=false, %% if true, add the section numbers to the bookmarks
    %pdfstartpage={1},        %% determines, on which page the PDF file is opened
    pdfpagemode=None         %% None, UseOutlines (=show bookmarks), UseThumbs (show thumbnails), FullScreen
  ]{hyperref}


  %%% provide all graphics (also) in this format, so you don't have
  %%% to add the file extensions to the \includegraphics-command
  %%% and/or you don't have to distinguish between generating
  %%% dvi/ps (through latex) and pdf (through pdflatex)
  \DeclareGraphicsExtensions{.pdf}

\else %else   (definitions for using latex instead of pdflatex)

  \usepackage[dvips]{graphicx}

  \DeclareGraphicsExtensions{.eps}

  \usepackage[%
    dvips,           %% sets up hyperref for use with the dvips driver
    colorlinks=false %% better for printout version; almost every hyperref-extension is eliminated by using dvips
  ]{hyperref}

\fi


%%% sets the PDF-Information options
\hypersetup{
  pdftitle={},
  pdfauthor={}, %%
  pdfsubject={}, %%
  pdfcreator={}, %%
  pdfproducer={}, %%
  pdfkeywords={} %%
}

%%%%%%%%%%%%%%%%%%%%%%%%%%%%%%%%%%%%%%%%%%%%%%%%%%%%%%%%%%%%%%%%%%%%%%%%%%%%%%%%
%%%
%%% define the titlepage
%%%

% \subject{}   %% subject which appears above titlehead
% \titlehead{} %% special heading for the titlepage

%%% title
\title{Intelligente Systeme Ausarbeitung}

%%% author(s)
\author{Florian Nehmer (Matr.Nr.: 2193399)}

%%% date
\date{\today}


%%%%%%%%%%%%%%%%%%%%%%%%%%%%%%%%%%%%%%%%%%%%%%%%%%%%%%%%%%%%%%%%%%%%%%%%%%%%%%%%
%%%
%%% set heading and footer
%%%

%%% scrheadings default:
%%%      footer - middle: page number
\pagestyle{scrheadings}


%%%%%%%%%%%%%%%%%%%%%%%%%%%%%%%%%%%%%%%%%%%%%%%%%%%%%%%%%%%%%%%%%%%%%%%%%%%%%%%%
%%%
%%% begin document
%%%

\begin{document}


%%% include the title
\maketitle
\thispagestyle{empty}

\newpage
\tableofcontents
\thispagestyle{empty}

\newpage
\pagenumbering{arabic}

%%%%%%%%%%%%%%%%%%%%%%%%%%%%%%%%%%%%%%%%%%%%%%%%%%%%%%%%%%%%%%%%%%%%%%%%%%%%%%%%
%%%
%%% begin main document
%%% structure: \section \subsection \subsubsection \paragraph \subparagraph
%%%

\section{Aufgabe Suchen, Lernen, NLP}

\subsection{Suchen}
\begin{itemize}
  \item Monte-Carlo-Tree-Search
  \item Exploration vs. Exploitation
  \item UCB1
  \item Eigenschaften des Algorithmus: Aheuristik, Asymmetrische Baumbildung, Game State Unabhängig
\end{itemize}

\subsection{Lernen}
\begin{itemize}
  \item Reinforcement Learning
  \item Markov Entscheidungsprozess
  \item Q-Value
  \item Deep Q-Learning
  \item Exploration vs. Exploitation
  \item Experience Replay
\end{itemize}

\subsection{NLP}
\begin{itemize}
  \item Rekurrente Neuronale Netze
  \item Aufbau und Funktionsweise
  \item Vorteile für NLP
  \item Probleme
\end{itemize}

\newpage

\section{Was macht Intelligenz aus?}

2014 prophezeite \textsc{Steven Hawking} der British Broadcasting Corporation (BBC) als Teil seiner Antwort auf eine Frage über sein Kommunikationsgerät, welches auf einem künstilchen Intelligenz (KI) System basierte, dass die Entwicklung voller künstlicher Intelligenz zukünftig das Ende der Menschheit bedeuten würde (vgl. \cite{BBC2014}).

Doch sind diese heute schon realen KI-Systeme die Vorläufer einer vollen künstlichen Intelligenz? Ist es überhaupt möglich, dass Maschinen mit einer vollen KI an die menschliche Intelligenz heranreichen? Und was macht eigentlich Intelligenz aus? Diesen Fragen wird der nachfolgende Text auf den Grund gehen. Um sich dem Begriff der Intelligenz zu nähern, werden zunächst verschiedene Intelligenzmodelle aus dem Bereich der Psychologie vorgestellt. Daraufhin beschäftigt sich der Text damit, was gegenwärtig unter dem Begriff ``Küstliche Intelligenz'' verstanden wird. Im Anschluss wird die menschliche Intelligenz der künstlichen Intelligenz gegenüber gestellt, um zu erörtern, was menschliche Intelligenz ausmacht. Es folgt ein Ausblick, was zukünftig mit Maschinen theoretisch möglich sein könnte. Abschließend werden die aufgestellten Thesen in einem Fazit zusammengefasst.

\subsection{Intelligenz}
Für den Begriff ``Intelligenz'' (von lateinisch intellegere - erkennen, einsehen, verstehen) existiert keine allgemeingültige Definition. In der Psychologie sind im Laufe der Zeit vielmehr verschiedene Intelligenzmodelle entstanden. Im Folgenden werden drei Intelligenzmodelle herangezogen, um verschiedene Facetten des Intelligenzbegriffes aufzuzeigen.

\subsubsection{Zwei-Faktoren-Modell von \textsc{Cattel}}
Zu den klassischen Theorien gehört insbesondere das Zwei-Faktoren-Modell von \textsc{Cattel}, welcher \textsc{Spearmans} Theorie (siehe auführlicher \cite{Spektrum2000}) präzisiert, indem er zwei Arten von Intelligenz präzisiert. Die \textit{Kristalline Intelligenz} einerseits, welche die Erfahrungen darstelle, die ein Mensch im Laufe seines Lebens sammle und die Fakten, die er dadurch erlerne. Somit sei die kristalline Intelligenz stark kulturell beeinflusst. Andererseits wird die \textit{Fluide Intelligenz} beschrieben, welche die Fähigkeit eines Menschen repräsentiere, sich Fakten und Erfahrungen anzueigenen. Ein hohes Maß an fluider Intelligenz sei notwendig, um sich schnell in unbekannten Situationen zurecht zu finden. Die fluide Intelligenz sei genetisch determiniert.  Sie nehme ab dem 25. Lebensjahr ab. Die kristalline Intelligenz hingegen steige bis zum 25. Lebenjahr stark an. Danach lasse sich nur noch ein langsamer Anstieg beobachten (vgl. \cite{Dorsch2019}). Es zeigen sich somit in diesem Modell schon zwei sehr grobe Facetten des Begriffes ``Intelligenz''.

\subsubsection{Multiple Intelligenzen nach \textsc{Gardner}}
Im Gegensatz zum Modell von \textsc{Cattel} dominieren in der jüngeren Forschung multidimensionale und prozessorientierte Modelle der Intelligenz. Hierbei wird Intelligenz als Sammlung von Fähigkeiten betrachtet und nicht wie bei \textsc{Cattel} als angeeignetes Wissen. So formuliert \textsc{Gardner} in seinem Model der \textit{multiplen Intelligenzen} zehn Dimensionen der Intelligenz. Die zehn Dimensionen beziehen auch Aspekte der menschlichen Wahrnehmung ein. Zur Definitionen dieser Dimensionen stellt \textsc{Garnder} verschiedene Kriterien auf. Eine intellektuelle Fähigkeit müsse einer bestimmten Hirnregion zugeordnet werden können, desweiteren müsse für diese Fähigkeiten eine Spezialbegabung nachgewiesen werden können. Andere Fähigkeiten hingegen könnten gleichzeitig durchschnittlich und/oder unterdurchschnittlich ausgeprägt sein. Außerdem sollte die Fähigkeit unabhängig von unterschiedlichen kulturellen Einflüssen sein. Auch müsse eine intellektuelle Fähigkeit expirementell nachweisbar und spezifisch geistig operationalisierbar sein. \textsc{Gardner} modelliert daher folgende Dimensionen der Intelligenz:
\begin{itemize}
  \item Linguistische Intelligenz erlaubt es Individuuen, miteinander über Sprache zu kommunizieren.
  \item Logisch-Mathematische Intelligenz erlaubt es Individuen, abstrakte Modelle und Relation zu benutzen bzw. zu erkennen.
  \item Musikalische Intelligenz erlaubt es Leuten, Bedeutungen durch Geräusche zu kreieren, zu verstehen oder über diese zu kommunizieren.
  \item Visuelle/Räumliche Intelligenz macht es möglich, dass Menschen Bilder aus ihrem Gedächtnis widerherstellen können und räumlich denken können.
  \item Körperlich Kinästhetische Intelligenz beschreibt die Fähigkeit, alle Teile des Körpers einzusetzen, um Probleme zu lösen. Diese Intelligenz sei beispielsweise bei Athleten oder Chirurgen ausgeprägt.
  \item Interpersonale Intelligenz ist auch als Empathie bekannt und bezeichnet die Fähigkeit, unausgesprochene Gefühle zu erkennen und zu beeinflussen.
  \item Intrapersonelle Intelligenz bezeichnet die Intelligenz, eigene Gefühle, Stimmungen und Schwächen zu erkennen und zu beinflussen.
  \item Naturalistische Intelligenz ist die Fähigkeit, Naturphänomene zu erkennen und eine Sensibilität für sie zu entwickeln. Diese Fähigkeit sei bei Naturforschern und Landwirten stark ausgeprägt.
\end{itemize}
Kürzlich wurden die Dimension um die Ökologisch-naturalistische, die spirituale und die Existenziale Dimension ergänzt (vgl. \cite{Gardner1993}).

Zusammenfassend kann man sagen, dass die Möglichkeiten der Wahrnehmung und die emotionale Intelligenz des Menschen eine wichtige Rolle bei \textsc{Gardners} Intelligenztheorie spielen. Weitere prozessorientierte Intelligenzmodelle, wie \textsc{Robert Sternbergs} \textit{triarchisches Intelligenzmodell} (vgl. \cite{Stern1984}) betonen hingegen eine stärkere soziokulturelle Abhängigkeit von intellektuellen Fähigkeiten. Desweiteren wird in der Forschung kontrovers diskutiert, welche Anteile der Intelligenz genetisch bedingt sind und welche Anteile die Sozialisation eines Menschen zu seinen intellektuellen Fähigkeiten beiträgt (Anlage-Umwelt-Kontroverse).

\subsection{Künstliche Intelligenz}
Im Folgenden wird der Begriff der künstlichen Intelligenz näher betrachtet. Auch hier gibt es verschiedene Ansätze, den Begriff zu definieren. Eine gängige Ansicht ist es, dass KI-Systeme verschiedene Grade an Intelligenz besitzen. So definiert \textsc{Mainzer} den Begriff ``Künstliche Intelligenz'' folgendermaßen: ``Ein System heißt intelligent, wenn es selbstständig und effizient Probleme lösen kann. Der Grad der Intelligenz hängt vom Grad der Selbstständigkeit, dem Grad der Komplexität des Problems und dem Grad der Effizienz des Problemlösungsverfahrens ab'' \cite{Mainzer2003}. Aus dieser Definition geht zudem hervor, dass künstliche Intelligenz sich in fast allen Fällen auf bestimmte Problemstellungen bezieht. Man bezeichnet solche KI-Systeme daher auch häufig als Expertensysteme. Dazu passend definierte bereits der amerikanische Informatiker \textsc{Elaine Rich} 1985 künstliche Intelligenz als ``The study of how to make computers do things that people are better at.'' (vgl. \cite{Rich1985})

Auf Grundlage der dargestellten Intelligenzmodelle kann nun der Begriff ``Maschinelles Lernen'' als Teilaspekt der künstlichen Intelligenz eingeordnet werden. Das maschinelle Lernen ist vergleichbar mit der fluiden Intelligenz von \textsc{Cattel}, denn maschinelles Lernen beschreibt den Lernprozess von Systemen (z.B. durch künstliche neuronale Netze). Der Unterschied zu der fluiden Intelligenz ist, dass die fluide Intelligenz des Menschen nicht auf einen spezifischen Bereich eingeschränkt sein muss. Wohingegen maschinelle Lernmethoden  Lösungsansätze für spezifische Problemstellungen erlernen. Da derzeit noch keine ``Allround'' maschinelle Lernmethode entwickelt wurde, welche Lösungen für unbekannte Probleme verschiedener Bereiche erlernen kann. Man versucht heutzutage also gezielt einzelne Probleme zu lösen, die ein Mensch derzeit besser lösen kann.

\subsection{Was Intelligenz ausmacht}
Mihlife der bis jetzt gewonnen Erkenntnisse kann nun skizziert werden, was Intelligenz ausmacht. Es wurde festgestellt, dass Intelligenz verschiedene Facetten hat. Es gibt sowohl die Intelligenz, die auf einen Erfahrungsschatz zurückgreifen kann, um in bekannten Situationen Entscheidungen zu treffen als auch die Intelligenz, die in unbekannten Situationen Probleme lösen kann. Intelligenz lässt sich zudem in verschiedene Dimensionen kategorisieren, die auch die Wahrnehmungsmechanismen und die unterbewussten kognitiven Prozesse des jeweiligen Systems mit einschließen. Das heißt, dass das volle Potential der Intelligenz erst ausgeschöpft werden kann, wenn auch diese Wahrnehmungsmechanismen voll ausgeprägt sind. Mechanismen wie Mitgefühl oder Empathie, also die hier nicht näher ausgeführte emotionale Intelligenz (siehe \cite{Mayer1990}), sind teilweise noch schwieriger zu erfassen, denn solche Empfindungen oder Gefühle setzten sich aus der Kombination von Wahrnehmungsprozesen und kognitiven Leistungen basierend auf Erfahrungen zusammen. Intelligenz setzt sich also aus einer Vielzahl von Faktoren zusammen und ist abhängig davon, wie das System seine Umwelt wahrnehmen kann bzw. was die Realität für dieses System ist. Denn diese Wahrnehmung hat einen direkten Einfluss darauf, wie die Intelligenz sich weiterentwickeln kann. Desweiteren gäbe es nach dem Modell von \textsc{Cattel}, wie oben erwähnt, auch angeborene Eigenschaften von Intelligenz, was wiederum in der Anlage-Umwelt-Kontroverse zum Ausdruck kommt. Die menschliche und die künstliche Intelligenz kann man demnach, auch nur vergleichen, wenn man in Betracht zieht, was die Maschine für eine Realität hat, also was die Maschine im Vergleich zum Menschen wahrnehmen kann.

\subsection{Was ist auf der Maschine möglich?}
Heutzutage sind KI-Systeme weitesgehend Expertensysteme, die darauf spezialisiert sind, bestimmte Probleme besser zu lösen als Menschen. Auch humanoide Roboter wie ATLAS von der Firma Boston Dynamics (siehe: \cite{Atlas2019}) sind Expertensysteme, die versuchen so gut wie möglich die menschliche Kinetik zu imitieren, und so bei den Beobachtern den Reiz auslösen, die eigene Kinetik mit der des Roboters zu vergleichen. Dadurch wird die Debatte verstärkt, dass die Maschine den Menschen künftig übertreffen könnte, es wird allerdings außer Acht gelassen, dass diese Maschine ihre spezifische Aufgabe nur erfüllt. Sie ist ein Experte für spezifische Bewegungen. Jedoch muss auch in Betracht gezogen werden, über welche Möglichkeiten diese Roboter verfügen bzw. was ihre Realität determiniert. Denn sie verfügen über eine Reihe von Sensoren, die es ihnen ausschließlich ermöglicht, sich selbst und einen Teil der Umwelt wahrzunehmen. Ihre Realität beschränkt sich also auf diese Sensorenwahrnehmung, welche allerdings nur einen kleinen Teil der meschlichen Sensorik imitiert. Mithilfe dieser Realität schaffen sie es, ein spezifisches Problem aus menschlicher Sicht nahezu optimal zu lösen.

Aber was ist nun, wenn die Wahrnehmung sowie die maschinellen Lernprozesse dieser Maschinen weiterhin verbessert werden? Das heißt, was wird in Zukunft mit intelligenten Maschinen möglich sein? Laut dem Physiker Max Tegmark ``Gibt es kein Gesetz der Physik, das besagt, dass man nicht auch Maschinen konstruieren kann, die in jeder Hinsicht intelligenter sind als wir'' \cite{Tegmark2017}. Er ist weiterhin der Meinung, dass das, ``was uns Menschen zu den dominanten Einheiten auf diesem Planeten macht, (...) nicht unsere Kraft, sondern unsere Intelligenz [ist]. Wenn wir Maschinen konstruieren, die schlauer sind als wir, dann gibt es deswegen keine Garantie dafür, dass wir die Kontrolle behalten werden'' (\cite{Tegmark2017}). Diese Ansicht geht mit dem Zitat von \textsc{Steven Hawking} einher, nur dass Tegmark präzisiert, dass die Entwicklung einer vollen künstllichen Intelligenz nicht das Ende der Menschheit bedeute, sondern es wichtig sei, wie man mit so einer Entwicklung umgehe. Eine solche künstliche Intelligenz könnte beispielsweise Arbeitsplätze redundant machen, welche dann von der Maschine besser und effizienter erledigt werden könnten. Eine derartige Entwicklung konnte in der Vergangenheit bereits beobachtet werden, als Maschinen zum Beispiel in der Automobilindustrie eingesetzt wurden, weil sie bestimmte körperliche Arbeiten ausdauernder und präziser durchführen konnten als Menschen. Wenn nun mehr geistige/intellektuelle Arbeitsplätze wegfallen, muss ebenso darauf geachtet werden, dass der wirtschaftliche Erfolg nicht bei einzelnen bleibt, die eine solche KI-Technologie bereitstellen, sondern die Gesellschaft davon profitiert.

\subsection{Fazit}
Es lässt sich zusammenfassen, dass die Intelligenz sich durch ihre Vielfalt auszeichnet. Kognitive Gegebenheiten reichen nicht aus, um einen hohen Grad an Intelligenz zu entwickeln, da ein Aspekt der Intelligenz ist, dass das intelligente System bei unbekannten Problemen diese analysieren und eine geeigente Lösung finden kann. Hierfür ist die wahrnehmbare Realität des Systems wichtig. Aus verschiedenen Wahrnehmungsmöglichkeiten lassen sich weitere Dimensionen der Intellligenz entwickeln wie die emotionale Intelligenz beim Menschen. Laut den Physikern \textsc{Steven Hawking} und \textsc{Max Tegmark} gibt es keine Faktoren, die die Entwicklung einer Intelligenz, die die Intelligenz eines Menschens übertrifft, stoppen würde, sodass es notwendig wird, dass die Menschheit vorher lernt damit umzugehen bzw. so eine Entwicklung unterbindet. Sie argumentieren, dass auch komplexe Dimensionen der Intelligenz wie Empathie oder Gefühle nur ein Zusammenspiel biologischer, kognitiver Prozesse sind, welche potentiell auf einer Maschine realisierbar sein können. Offen bleibt, zu welchem Zeitpunkt eine solche Entwicklung möglich ist und welche Strategien die Menschheit entwickelt, um mit dieser Entwicklung umzugehen.

\newpage



\section{Was kann ich tun, um als Informatiker verantwortlich zu handeln?}
Um als Informatiker verantwortungsbewusst zu handeln, kann ich mir bei jeder Konzeption/Entwicklung und Erforschung von Systemen und informationstechnischen Konzepten folgende Aspekte verinnerlichen und umsetzen.
\begin{itemize}
  \item Ich bin für meine eigenen Entscheidungen verantwortlich. Mir ist bewusst, dass Informationstechnologie schon in den Großteil der Lebensbereiche eingedrungen ist und ich mir Gedanken über die Auswirkung meiner Entwicklung machen muss.
  \item Das Vertrauen, welches mir als Fachexperte entgegen gebracht wird, nutze ich nicht aus und kläre über mögliche Gefahren von Informationssystemen auf.
  \item Ich lehne die Konzeption und Entwicklung von Informationssystemen ab, die Menschen- oder Bürgerrechte verletzen.
  \item Ich sammle für meine Systeme teilweise persönliche Daten von Menschen. Diese muss ich schützen und nur so lange speichern, wie es für mein System notwendig ist. Ich behandle die Daten also so, als wären es meine eigenen.
  \item Die zunehmende Zahl an technischen Geräten sorgt für eine größere Nutzung von Energie und Ressourcen. Ich versuche mit Informationstechnologie, keine Energien oder Ressourcen zu verschwenden.
\end{itemize}

Diese Aspekte stammen aus dem Manifest für verantwortungsbewusste Softwareentwicklung (\cite{Loewe2015}) und lassen sich meiner Meinung nach auf alle Teilbereiche der Informatik beziehen. Diese kann ich unterschreiben, um mit meinem Namen hinter ihnen zu stehen.

\clearpage
\pagestyle{plain}

\newpage


\printbibliography

\addcontentsline{toc}{section}{Literaturverzeichnis}














%%%
%%% end main document
%%%
%%%%%%%%%%%%%%%%%%%%%%%%%%%%%%%%%%%%%%%%%%%%%%%%%%%%%%%%%%%%%%%%%%%%%%%%%%%%%%%%

% \appendix  %% include it, if something (bibliography, index, ...) follows below

%%%%%%%%%%%%%%%%%%%%%%%%%%%%%%%%%%%%%%%%%%%%%%%%%%%%%%%%%%%%%%%%%%%%%%%%%%%%%%%%
%%%
%%% bibliography
%%%
%%% available styles: abbrv, acm, alpha, apalike, ieeetr, plain, siam, unsrt
%%%
% \bibliographystyle{plain}

%%% name of the bibliography file without .bib
%%% e.g.: literatur.bib -> \bibliography{literatur}
% \bibliography{FIXXME}

\end{document}
