\documentclass[a4paper, 11pt]{scrartcl}

\usepackage{ngerman}

\usepackage[utf8]{inputenc}

\usepackage[T1]{fontenc}

\usepackage{ae,aecompl}

\usepackage{amsmath,amssymb,amstext}

\usepackage{psfrag}

\usepackage{biblatex}

\bibliography{sources.bib}

\usepackage[automark]{scrpage2}

\usepackage{ifpdf}

\ifpdf%   (definitions for using pdflatex instead of latex)

  %%% graphicx: support for graphics
  \usepackage[pdftex]{graphicx}

  \pdfcompresslevel=9

  %%% hyperref (hyperlinks in PDF): for more options or more detailed
  %%%          explanations, see the documentation of the hyperref-package
  \usepackage[%
    %%% general options
    pdftex=true,      %% sets up hyperref for use with the pdftex program
    %plainpages=false, %% set it to false, if pdflatex complains: ``destination with same identifier already exists''
    %
    %%% extension options
    backref,      %% adds a backlink text to the end of each item in the bibliography
    pagebackref=false, %% if true, creates backward references as a list of page numbers in the bibliography
    colorlinks=true,   %% turn on colored links (true is better for on-screen reading, false is better for printout versions)
    %
    %%% PDF-specific display options
    bookmarks=true,          %% if true, generate PDF bookmarks (requires two passes of pdflatex)
    bookmarksopen=false,     %% if true, show all PDF bookmarks expanded
    bookmarksnumbered=false, %% if true, add the section numbers to the bookmarks
    %pdfstartpage={1},        %% determines, on which page the PDF file is opened
    pdfpagemode=None         %% None, UseOutlines (=show bookmarks), UseThumbs (show thumbnails), FullScreen
  ]{hyperref}


  %%% provide all graphics (also) in this format, so you don't have
  %%% to add the file extensions to the \includegraphics-command
  %%% and/or you don't have to distinguish between generating
  %%% dvi/ps (through latex) and pdf (through pdflatex)
  \DeclareGraphicsExtensions{.pdf}

\else %else   (definitions for using latex instead of pdflatex)

  \usepackage[dvips]{graphicx}

  \DeclareGraphicsExtensions{.eps}

  \usepackage[%
    dvips,           %% sets up hyperref for use with the dvips driver
    colorlinks=false %% better for printout version; almost every hyperref-extension is eliminated by using dvips
  ]{hyperref}

\fi


%%% sets the PDF-Information options
\hypersetup{
  pdftitle={},
  pdfauthor={}, %%
  pdfsubject={}, %%
  pdfcreator={Accomplished with LaTeX2e and pdfLaTeX with hyperref-package.}, %%
  pdfproducer={}, %%
  pdfkeywords={} %%
}

%%%%%%%%%%%%%%%%%%%%%%%%%%%%%%%%%%%%%%%%%%%%%%%%%%%%%%%%%%%%%%%%%%%%%%%%%%%%%%%%
%%%
%%% define the titlepage
%%%

% \subject{}   %% subject which appears above titlehead
% \titlehead{} %% special heading for the titlepage

%%% title
\title{Intelligente Systeme Ausarbeitung}

%%% author(s)
\author{Florian Nehmer (Matr.Nr.: 2193399)}

%%% date
\date{\today}


%%%%%%%%%%%%%%%%%%%%%%%%%%%%%%%%%%%%%%%%%%%%%%%%%%%%%%%%%%%%%%%%%%%%%%%%%%%%%%%%
%%%
%%% set heading and footer
%%%

%%% scrheadings default:
%%%      footer - middle: page number
\pagestyle{scrheadings}


%%%%%%%%%%%%%%%%%%%%%%%%%%%%%%%%%%%%%%%%%%%%%%%%%%%%%%%%%%%%%%%%%%%%%%%%%%%%%%%%
%%%
%%% begin document
%%%

\begin{document}

%%% include the title
\maketitle

\newpage
\tableofcontents

\newpage

%%%%%%%%%%%%%%%%%%%%%%%%%%%%%%%%%%%%%%%%%%%%%%%%%%%%%%%%%%%%%%%%%%%%%%%%%%%%%%%%
%%%
%%% begin main document
%%% structure: \section \subsection \subsubsection \paragraph \subparagraph
%%%

\section{Aufgabe Suchen, Lernen, NLP; Jeweils $\frac{1}{2}$ Seite}
noch ein Hinweis zu Ihren Ausarbeitungen: Für 'Suchen', 'Lernen' und 'Verarbeitung natürlicher Sprache' soll nur eine Stichwortliste
abgegeben werden, bitte kein Fließtext. Es geht darum, dass Sie sich selbst klar machen, welche Konzepte Sie sich erarbeitet haben.
Ein Beispiel wie so etwas aussehen kann für 'Suchen'

\begin{itemize}
\item A*-Algorithmus im Detail
\item IDA*-Algorithmus im Detail
\item Zulässigkeit
\item Monotonie
\item Heuristiken
\item Eigenschaften der beiden Algorithmen: Vollständigkeit, Optimalität
\end{itemize}

\newpage

\section{Intelligenz; Max. 5 Seiten}
Argumentation auf Basis von Literatur: Was macht Intelligenz aus und was ist auf einer
Maschine möglich?

\newpage

\section{Was kann ich tun, um als Informatiker verantwortlich zu handeln?; $\frac{1}{2}$ Seite}

\cite{fitzgerald:realigning_research_and_practice}












%%%
%%% end main document
%%%
%%%%%%%%%%%%%%%%%%%%%%%%%%%%%%%%%%%%%%%%%%%%%%%%%%%%%%%%%%%%%%%%%%%%%%%%%%%%%%%%

% \appendix  %% include it, if something (bibliography, index, ...) follows below

%%%%%%%%%%%%%%%%%%%%%%%%%%%%%%%%%%%%%%%%%%%%%%%%%%%%%%%%%%%%%%%%%%%%%%%%%%%%%%%%
%%%
%%% bibliography
%%%
%%% available styles: abbrv, acm, alpha, apalike, ieeetr, plain, siam, unsrt
%%%
% \bibliographystyle{plain}

%%% name of the bibliography file without .bib
%%% e.g.: literatur.bib -> \bibliography{literatur}
% \bibliography{FIXXME}

\end{document}
